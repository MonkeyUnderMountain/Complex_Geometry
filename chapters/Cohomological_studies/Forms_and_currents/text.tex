\section{Forms and Currents}

    Let \(M\) be a complex manifold of complex dimension \(d\).


\subsection{Differential forms}

    Recall that we have the decomposition of the cotangent bundle:
    \[ \Omega_{\sm}^{1,\bbC} \cong \Omega_{\sm}^{1,0} \oplus \Omega_{\sm}^{0,1} \cong \Omega_{\sm}^1 \oplus \overline{\Omega_{\sm}^1}. \]
    Take exterior powers, we have\( \Omega_{\sm}^{k,\bbC} \cong \bigoplus_{p+q=k} \Omega_{\sm}^{p,q} \),
    where \(\Omega_{\sm}^{p,q} = \bigwedge^p \Omega_{\sm}^{1,0} \otimes \bigwedge^q \Omega_{\sm}^{0,1}\).
    We also use the notation
    \[ \calA^{p,q} := \Omega_{\sm}^{p,q}, \quad \calA^{k} := \Omega_{\sm}^{k,\bbC}. \]
    
    A reason to induce this strange sheaf \(\calA^{k} = \Omega_{\sm}^{k,\bbC}\) is to make sense of integration of top-degree forms.
    For simplicity, assume that \(M\) is compact.
    Let \(\omega \in \calA^{2d}(M)\) be a smooth complex-valued \(2d\)-form on \(M\).
    Then its integration is well-defined in the smooth manifold sense:
    \[ \int_M \omega \in \bbC. \]
    However, in complex case, it is more natural to ``integral'' a holomorphic \(d\)-form on a \(d\)-dimensional complex manifold.
    This does not make sense in the smooth manifold theory.
    The solution is to associate a holomorphic \(d\)-form \(\eta \in \calA^{d,0}(M)\) with a smooth \(2d\)-form (\((d,d)\)-form) \(\omega = \eta \wedge \overline{\eta} \in \calA^{d,d}(M) \subset \calA^{2d}(M)\).

    Another reason  is that \(\bigoplus_k \Omega_{\sm}^{k}\) is not closed under the exterior derivative \(\upd\), while \(\bigoplus_k \Omega_{\sm}^{k,\bbC}\) is. 
    Suppose that we have local holomorphic coordinates \((z_1, \ldots, z_d)\).
    Recall that we have the exterior derivative
    \[ \upd: \Omega_{\sm}^{1,0} \to \Omega_{\sm}^{2,\bbC}, \quad f d z_{i} \mapsto \sum_{j=1}^d \frac{\partial f}{\partial z_j} d z_j \wedge d z_{i} + \sum_{i=1}^d \frac{\partial f}{\partial \overline{z_j}} d \overline{z_j} \wedge d z_{i}. \]
    On its conjugation, we have
    \[ \upd: \Omega_{\sm}^{0,1} \to \Omega_{\sm}^{2,\bbC}, \quad f d \overline{z_{i}} \mapsto \sum_{j=1}^d \frac{\partial f}{\partial z_j} d z_j \wedge d \overline{z_{j}} + \sum_{j=1}^d \frac{\partial f}{\partial \overline{z_j}} d \overline{z_j} \wedge d \overline{z_{j}}. \]

    Extending \(\upd\) by linearity and the Leibniz rule \(\upd(\alpha \wedge \beta) = \upd \alpha \wedge \beta + (-1)^{\deg \alpha} \alpha \wedge \upd \beta\), we get the exterior derivative
    \[ \upd: \calA^k = \Omega_{\sm}^{k,\bbC} \to \calA^{k+1} = \Omega_{\sm}^{k+1,\bbC}, \]
    which can be decomposed as \(\upd = \partial + \overline{\partial}\), where 
    \begin{align*}
        \partial: \calA^{p,q} &\to \calA^{p+1,q}, \quad f d z_I \wedge d \overline{z_J} \mapsto \sum_{j=1}^d \frac{\partial f}{\partial z_j} d z_j \wedge d z_I \wedge d \overline{z_J}, \\
        \overline{\partial}: \calA^{p,q} &\to \calA^{p,q+1}, \quad f d z_I \wedge d \overline{z_J} \mapsto \sum_{j=1}^d \frac{\partial f}{\partial \overline{z_j}} d \overline{z_j} \wedge d z_I \wedge d \overline{z_J}.
    \end{align*}
    In a diagram, we have:
    \begin{center}
        \begin{tikzpicture}
            \matrix (m) [matrix of math nodes, column sep=2em, row sep=2em] {
                \bullet & \bullet & \bullet & \bullet & \bullet \\
                \bullet & \Omega^{p,q+1}  & \bullet & \bullet & \bullet \\
                \bullet & \Omega^{p,q} & \Omega^{p+1,q} & \bullet & \bullet \\
                \bullet & \bullet & \bullet & \bullet & \bullet \\
                    &  &  & \Omega^{k} & \Omega^{k+1} \\
                };
                % Draw a horizontal line before the row containing \Omega^k (above row 5)
                \draw (m-4-1.south) -- (m-4-5.south);
                % Draw a dashed line passing through \Omega^{p-1,q+1} and \Omega^k
                \draw[dashed] (m-2-1) -- (m-5-4);
                \draw[dashed] (m-1-1) -- (m-5-5);
                % Draw arrows for the operators
                \draw[->] (m-3-2) -- (m-3-3) node[midway, above] {\(\partial\)};
                \draw[->] (m-3-2) -- (m-2-2) node[midway, right] {\(\overline{\partial}\)};
                % \draw[->] (m-3-2) -- (m-4-4) node[midway, right] {\(\mu\)};
                % \draw[->] (m-3-2) -- (m-1-1) node[midway, above] {\(\overline{\mu}\)};
                \draw[->] (m-5-4) -- (m-5-5) node[midway, above] {\(\upd\)};
        \end{tikzpicture} 
    \end{center}

    \begin{proposition}\label{prop:partial_and_partial_bar_are_differential_operators}
        The operators \(\partial\) and \(\overline{\partial}\) satisfy
        \[ \partial^2 = 0, \quad \overline{\partial}^2 = 0, \quad \partial \overline{\partial} + \overline{\partial} \partial = 0. \]
    \end{proposition}
    \begin{proof}
        \Yang{To be added.}
    \end{proof}

    \begin{proposition}\label{prop:holomorphic_pull_back_compatiable_with_A^pq}
        Let \(f: M \to N\) be a holomorphic map between complex manifolds.
        Then the pull-back of differential forms \(f^*: \calA^{k}_N \to \calA^{k}_M\) satisfies
        \[ f^*(\calA^{p,q}_N) \subset \calA^{p,q}_M, \quad f^* \circ \partial_N = \partial_M \circ f^*, \quad f^* \circ \overline{\partial}_N = \overline{\partial}_M \circ f^*. \]
    \end{proposition}
    \begin{proof}
        \Yang{To be added.}
    \end{proof}

    % \Yang{The following need to checked.}
    % \paragraph{Topological vector space of forms with compact support}
    % Let \(M\) be a complex manifold of complex dimension \(d\).
    \begin{construction}
        % We denote by \(\calA_c^{k}(M)\) (resp. \(\calA_c^{p,q}(M)\)) the space of smooth complex-valued \(k\)-forms (resp. \((p,q)\)-forms) on \(M\) with compact support.
        To define a topology on \(\calA^{k}(M)\), we use the collection of seminorms as follows.
        Let \(U \subset M\) be a coordinate chart with local real coordinates \((x_1, \ldots, x_{2d})\).
        Given \(\omega \in \calA^{k}(M)\), we can write \(\omega|_U\) as \(\omega|_U = \sum_I f_I d x_I\).
        For any multi-index \(\alpha = (\alpha_1, \ldots, \alpha_{2d})\), we can define the partial derivative
        \[ D^{\alpha} f = \frac{\partial^{|\alpha|} f}{\partial x_1^{\alpha_1} \cdots \partial x_{2d}^{\alpha_{2d}}}. \]
        Then for any compact subset \(K \subset U\) and non-negative integer \(m\), we can define a seminorm on \(\calA^{k}(M)\) by
        \[ p_{U, K, m}(\omega) = \sup_{x \in K} \max_{I} \max_{|\alpha| \leq m} |D^{\alpha} f_I(x)|, \]
        where \(m\) is a non-negative integer.

        The topology on \(\calA^{k}(M)\) is defined to be the weakest topology such that all seminorms \(p_{U, K, m}\) are continuous, where \(U\) runs over all coordinate charts of \(M\), \(K\) runs over all compact subsets of \(U\), and \(m\) runs over all non-negative integers.
    \end{construction}
    
    \begin{proposition}\label{prop:pullback_of_forms_is_continuous}
        Let \(f: M \to N\) be a holomorphic map between complex manifolds.
        Then the pull-back of differential forms \(f^*: \calA^{k}(N) \to \calA^{k}(M)\) is a continuous linear map with respect to the topologies defined above.
        \Yang{To be checked}
    \end{proposition}

    \begin{proposition}\label{prop:A^k_is_Fr\'echet_space}
        The space \(\calA^{k}(M)\) equipped with the topology defined above is a Fr\'echet space,
        i.e., a complete metrizable locally convex topological vector space.
        \Yang{To be checked}
    \end{proposition}


\subsection{Currents}

    We denote by \(\calA_c^{k}(M)\) (resp. \(\calA_c^{p,q}(M)\)) the space of smooth complex-valued \(k\)-forms (resp. \((p,q)\)-forms) on \(M\) with compact support.
    It is also denoted by \(\calD^{k}(M)\).

    \begin{definition}\label{def:currents}
        A \emph{current} of dimension \(k\) or degree \((2d - k)\) on \(M\) is a continuous linear functional
        \[ T: \calD^k(M) \to \bbC. \]
        We denote the space of currents of dimension \(k\) on \(M\) by \(\calD_k'(M)\) or \(\calD'^{2d-k}(M)\).
        % \Yang{To be revised.}
    \end{definition}