\section{Metrics, curvature and connections}

% \Yang{we want discuss the following topics here: metric, forms, connection, curvature, Chern classes, K\"ahler condition, examples.}

\subsection{The first properties}

    Let \(X\) be a complex manifold and \(E \to X\) a holomorphic vector bundle. 

    \begin{definition}\label{def:Hermitian_metric_on_holomorphic_vector_bundles}
        A \emph{Hermitian metric} on \(E\) is a smoothly varying family of Hermitian inner products \(\langle \cdot, \cdot \rangle_x\) on the fibers \(E_x\) for each \(x \in X\),
        i.e., 
        \[\langle \cdot, \cdot \rangle_x: E_x \times E_x \to \mathbb{C}\]
        is a Hermitian inner product for each \(x\), and for any local smooth sections \(s,t\) of \(E\), the function
        \[x \mapsto \langle s(x), t(x) \rangle_x\]
        is smooth on \(X\).
        % \Yang{To be continued.}
    \end{definition}

    \begin{definition}\label{def:hermitian_metric_on_tangent_bundle}
        A \emph{Hermitian metric} on a complex manifold \(X\) is a Hermitian metric on its holomorphic tangent bundle \(TX\).
    \end{definition}

    \begin{remark}\label{rmk:Riemannian_metric_induced_by_hermitian_metric}
        Let \(h\) be a Hermitian metric on a complex manifold \(X\).
        Then \(h\) induces a Riemannian metric \(g\) on the underlying real manifold of \(X\) by
        \[g(u,v) = \Real(h(u,v))\]
        for real tangent vectors \(u,v \in T_x X\).
        \Yang{To be revised.}
    \end{remark}

    \begin{example}\label{eg:Fubini-Study_metric_on_P^n}
        Let \(\bbP^n\) be the complex projective space.
        The \emph{Fubini-Study metric} \(h_{\text{FS}}\) is a Hermitian metric on its tangent bundle \(T \bbP^n\) defined as follows.
        On the standard affine chart \(U_i = \{[z_0: \cdots : z_n] \in \bbP^n \mid z_i \neq 0\}\) with coordinates \(z_{j,i} = z_j/z_i\) for \(j \neq i\), 
        we know that \(T\bbP^n|_{U_i}\) is spanned by \(\{\partial_{j,i} = \partial/\partial z_{j,i}\}_{j \neq i}\).
        The Fubini-Study metric is given by
        \[h_{\text{FS}}(z_{-,i})(\partial_k, \partial_l) = \frac{\delta_{kl}}{(1 + \sum_{r \neq i} |z_{r,i}|^2)} - \frac{\overline{z_{k,i}} z_{l,i}}{(1 + \sum_{r \neq i} |z_{r,i}|^2)^2}.\]

        On \(U_{ij} = U_i \cap U_j\), the differential form transform as
        \[ \upd z_{k,i} = z_{j,i} \upd z_{k,j} - z_{k,i} z_{j,i} \upd z_{i,j}.\]
        In the matrix form,
        \[\begin{bmatrix}
            \upd z_{1,i} \\
            \vdots \\   
            \upd z_{n,i}
        \end{bmatrix} =
        \begin{bmatrix}
            z_{j,i} & 0 & \cdots & -z_{1,i} z_{j,i} & \cdots & 0 \\
            0 & z_{j,i} & \cdots & -z_{2,i} z_{j,i} & \cdots & 0 \\
            \vdots & \vdots & \ddots & \vdots & \ddots & \vdots \\
            0 & 0 & \cdots & -z_{n,i} z_{j,i} & \cdots & z_{j,i}
        \end{bmatrix}
        \begin{bmatrix}
            \upd z_{1,j} \\
            \vdots \\   
            \upd z_{n,j}
        \end{bmatrix}.\]

        hence the tangent vectors transform as
        \[\partial_{k,i} = \frac{\partial}{\partial z_{k,i}} = z_{j,i} \partial_{k,j} \quad \text{for } k \neq j, \quad \text{and} \quad \]
        
        % tangent vectors transform as
        % \[\partial_{k,i} = \frac{\partial}{\partial (z_{k,j} z_{j,i})} = \frac{1}{z_{j,i}} \partial_{k,j} \]
        % One can check that the above definition is consistent on the overlaps \(U_{ij}\),
        \Yang{To be continued.}
    \end{example}

    \begin{example}\label{eg:Fubini-Study_metric_on_P^2}
        Now let us consider the complex projective plane \(\bbP^2 = \{[X:Y:Z]\}\).
        On the affine chart \(U_Z = \{[X:Y:Z] \mid Z \neq 0\}\) with coordinates \(x = X/Z\) and \(y = Y/Z\), 
        the Fubini-Study metric \(h_{\text{FS}}\) on \(T \bbP^2|_{U_Z}\) is given by
        \[ h_{\text{FS}}(x,y) = \frac{1}{(1 + |x|^2 + |y|^2)^2}\begin{bmatrix}
            1 + |y|^2 & -\overline{x} y \\
            -x \overline{y} & 1 + |x|^2
        \end{bmatrix}. \]
        For a tangent vector \(\partial = a \partial_x + b \partial_y\), its norm squared is
        \[ \|\partial\|_{h_{\text{FS}}}^2 = \frac{(1 + |y|^2)|a|^2 + (1 + |x|^2)|b|^2 - 2 \Real(\overline{x} y a \overline{b})}{(1 + |x|^2 + |y|^2)^2} = \frac{|a|^2 + |b|^2 + |x b - y a|^2}{(1 + |x|^2 + |y|^2)^2} \geq 0. \]
    \end{example}

    \begin{definition}\label{def:connection_on_holomorphic_vector_bundles}
        Let \(X\) be a complex manifold and \(E \to X\) a holomorphic vector bundle.
        A \emph{connection} on \(E\) is a \(\bbC\)-linear map between the sheaves of smooth sections
        \[\nabla: \calC^\infty(-, E) \to \calC^\infty(-, T^*X \otimes E)\]
        satisfying the Leibniz rule:
        \[\nabla(fs) = \upd f \otimes s + f \nabla s\]
        for all smooth functions \(f\) and smooth sections \(s\) of \(E\).
    \end{definition}

    When you choose a vector field \(v \in \calC^\infty(U, TX)\) on an open set \(U \subset X\), the connection \(\nabla\) induces an endomorphism
    \[\nabla_v: \calC^\infty(U, E) \to \calC^\infty(U, E)\]
    by applying \(v\) on the \(T^*X\) component of \(\nabla s\) for a section \(s\) of \(E\).
    In particular, if \(E = TX\) is the tangent bundle, then \(\nabla_v\) is called a \emph{covariant derivative} along \(v\).
    Sometimes people call \(\nabla\) an \emph{endomorphism-valued 1-form} on \(X\) with values in \(\End(E)\) by viewing it as a map \(v \mapsto \nabla_v\).

    \begin{example}\label{eg:connection_on_P^n}
        Let \(\mathbb{P}^n\) be the complex projective space and \(\mathcal{O}_{\mathbb{P}^n}(1)\) the hyperplane line bundle.
        The \emph{Chern connection} associated with the Fubini-Study metric on \(\mathcal{O}_{\mathbb{P}^n}(1)\) is a connection defined as follows:
        For a section \(s\) of \(\mathcal{O}_{\mathbb{P}^n}(1)\), we define
        \[\nabla s = ds + \alpha s,\]
        where \(\alpha\) is a (1,0)-form determined by the Fubini-Study metric.
        \Yang{To be continued.}
    \end{example}

    \begin{proposition}\label{prop:unique_compatible_connection_on_hermitian_vector_bundle}
        Let \(X\) be a complex manifold, \(E \to X\) a holomorphic vector bundle equipped with a Hermitian metric \(h\).
        Then there exists a unique connection \(\nabla\) on \(E\) that is compatible with both the holomorphic structure and the Hermitian metric \(h\).
        \Yang{To be checked.}
    \end{proposition}

    By the Leibniz rule, the connection \(\nabla\) can be extended to act on \(E\)-valued differential forms:
    \[\nabla: \calC^\infty(-, \Lambda^k T^*X \otimes E) \to \calC^\infty(-, \Lambda^{k+1} T^*X \otimes E)\]
    for all \(k \geq 0\), satisfying
    \[\nabla(\omega \otimes s) = \upd\omega \otimes s + (-1)^k \omega \wedge \nabla s\]
    for \(\omega \in \calC^\infty(-, \Lambda^k T^*X)\) and \(s \in \calC^\infty(-, E)\).

    \begin{definition}\label{def:curvature_of_connection}
        Let \(X\) be a complex manifold, \(E \to X\) a holomorphic vector bundle, and \(\nabla\) a connection on \(E\).
        The \emph{curvature} of the connection \(\nabla\) is defined as the endomorphism-valued 2-form
        \[F_\nabla = \nabla^2: \calC^\infty(-, E) \to \calC^\infty(-, \Lambda^2 T^*X \otimes E),\]
        where \(\nabla^2\) is the composition of \(\nabla\) with itself.
        \Yang{To be continued.}
    \end{definition}

    When \(E = TX\) is the tangent bundle, the curvature \(F_\nabla\) is a \((3,1)\)-tensor, which is the classical Riemann curvature tensor.


    \Yang{For a line bundle, everything coincide.}


\subsection{Chern-Weil Theory}

    \begin{theorem}\label{thm:Chern-Weil_theorem}
        Let \(X\) be a complex manifold and \(E \to X\) a holomorphic vector bundle equipped with a Hermitian metric \(h\).
        Let \(\nabla\) be the unique connection on \(E\) compatible with both the holomorphic structure and the Hermitian metric \(h\), and let \(F_\nabla\) be its curvature.
        Then the Chern classes \(c_k(E) \in H^{2k}(X, \mathbb{R})\) can be represented by the differential forms
        \[c_k(E) = \left[\frac{1}{(2\pi i)^k} \mathrm{Tr} \left( F_\nabla^k \right) \right].\]
        \Yang{To be checked.}
    \end{theorem}
    \begin{proof}
        \Yang{To be added.}
    \end{proof}


% \subsection{the K\"ahler condition}

%     \begin{definition}\label{def:Kahler_manifold}
%         A \emph{K\"ahler manifold} is a complex manifold \(X\) equipped with a Hermitian metric \(h\) whose associated (1,1)-form \(\omega\) is closed, i.e., \(d\omega = 0\).
%         \Yang{To be checked.}
%     \end{definition}