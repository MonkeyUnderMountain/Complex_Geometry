\section{Complex Manifolds}

\subsection{Definition and Examples}

    \begin{definition}\label{def:complex_manifold}
        A \emph{complex manifold} of complex dimension \(n\) is a topological space \(M\) such that
        \begin{enumerate}
            \item \(M\) is Hausdorff and second countable;
            \item \(M\) is locally homeomorphic to \(\bbC^n\), i.e., for every point \(p\in M\), there exists an open neighborhood \(U\) of \(p\) and a homeomorphism \(\varphi: U\to V\subset \bbC^n\), where \(V\) is an open subset of \(\bbC^n\),
                The pair \((U,\varphi)\) is called a \emph{chart};
            \item if \((U,\varphi)\) and \((U',\varphi')\) are two charts with \(U\cap U'\neq \emptyset\), then the transition map
                \[
                    \varphi'\circ \varphi^{-1}: \varphi(U\cap U')\to \varphi'(U\cap U')
                \]
                is holomorphic.
        \end{enumerate}
        The collection of all charts \(\{(U_\alpha,\varphi_\alpha)\}\) that cover \(M\) is called an \emph{atlas}.
        If the atlas is maximal, it is called a \emph{complex structure} on \(M\).
    \end{definition}

    Another way to define complex manifolds is to use the language of ringed spaces.

    \begin{definition}\label{def:complex_manifold_as_ringed_space}
        A \emph{complex manifold} of complex dimension \(n\) is a locally ringed space \((M,\calO_M)\) such that
        \begin{enumerate}
            \item \(M\) is Hausdorff and second countable;
            \item for every point \(p\in M\), there exists an open neighborhood \(U\) of \(p\) such that \((U,\calO_M|_U)\) is isomorphic to \((B,\calO_B)\), where \(B\) is the unit open ball in \(\bbC^n\) and \(\calO_B\) is the sheaf of holomorphic functions on \(B\).
        \end{enumerate}
    \end{definition}

    \begin{question}\label{qs:existence_and_uniqueness_of_complex_structure}
        Given a topological space \(M\) that is Hausdorff and second countable, when does it admit a complex structure?
        Is such a complex structure unique?
    \end{question}

    For complex dimension \(1\), the answer is positive and well-known.
    For higher dimensions, the answer is negative in general.
    In particular, does the \(6\)-sphere \(S^6\) admit a complex structure?
    This is a famous open problem in complex geometry.

    \begin{question}\label{qs:complex_structure_on_S6}
        Does the \(6\)-sphere \(S^6\) admit a complex structure?
    \end{question}

    \begin{definition}\label{def:holomorphic_map_between_complex_manifolds}
        Let \(M\) and \(N\) be two complex manifolds.
        A continuous map \(f: M\to N\) is called \emph{holomorphic} if for every point \(p\in M\), there exist charts \((U,\varphi)\) of \(M\) around \(p\) and \((V,\psi)\) of \(N\) around \(f(p)\) with \(U \subset f^{-1}(V)\) such that
        \[
            \psi\circ f\circ \varphi^{-1}: \varphi(U)\to \psi(V)
        \]
        is holomorphic.
    \end{definition}

    \begin{definition}\label{def:submanifold_of_complex_manifold}
        Let \(M\) be a complex manifold of complex dimension \(n\).
        A subset \(S\subset M\) is called a \emph{complex submanifold} of complex dimension \(k\) if for every point \(p\in S\), there exist a chart \((U,\varphi)\) of \(M\) around \(p\) such that
        \[
            \varphi(U\cap S) = \varphi(U)\cap (\bbC^k\times \{0\}) \subset \bbC^n,
        \]
        where we identify \(\bbC^n\) with \(\bbC^k\times \bbC^{n-k}\).
        This gives a chart of \(S\) around \(p\).
        Endowed with the induced topology and the induced complex structure, \(S\) is a complex manifold of complex dimension \(k\).
    \end{definition}

    \begin{example}\label{eg:complex_vector_space_as_complex_manifold}
        Any complex vector space \(V\) of complex dimension \(n\) is a complex manifold of complex dimension \(n\).        
    \end{example}

    \begin{example}\label{eg:complex_projective_space_as_complex_manifold}
        The complex projective space \(\bbC\bbP^n:= \bbC^{n+1}\setminus \{0\}/\bbC^\times\) is a complex manifold of complex dimension \(n\).
        In fact, \(\bbC\bbP^n\) can be covered by \(n+1\) charts, each of which is biholomorphic to \(\bbC^n\).
        For example, the chart \(U_0 = \{[z_0: z_1: \cdots: z_n]\in \bbC\bbP^n: z_0\neq 0\}\) is biholomorphic to \(\bbC^n\) via the map
        \[
            [z_0: z_1: \cdots: z_n] \mapsto \left(\frac{z_1}{z_0}, \frac{z_2}{z_0}, \ldots, \frac{z_n}{z_0}\right).
        \]
        The other charts are defined similarly.
    \end{example}

    \begin{proposition}\label{prop:complex_submanifold_by_holomorphic_implicit_function_theorem}
        Let \(M\) and \(N\) are complex manifolds of complex dimension \(n\) and \(m\) respectively, with \(n \geq m\).
        If \(f: M\to N\) is a holomorphic map such that \(p\) is a regular value of \(f\), 
        i.e., the tangent map \(\upd f_x\) is surjective for every \(x\in f^{-1}(p)\),
        then \(f^{-1}(p)\) is a complex submanifold of \(M\) of complex dimension \(n - m\).
    \end{proposition}
    \begin{proof}
        For every point \(q\in f^{-1}(p)\), choose charts \((U,\varphi)\) of \(M\) around \(q\) and \((V,\psi)\) of \(N\) around \(p\) such that \(f(U) \subset V\).
        By changing coordinates if necessary, we may assume that \(\det(\partial f/\partial w)(q) \neq 0\),
        where we write the coordinates of \(\varphi(U)\) as \((z,w) = (z_1, \ldots, z_{n-m}, w_1, \ldots, w_m) \in \bbC^{n-m} \times \bbC^m\).
        Then by the Holomorphic Implicit Function Theorem (\cref{thm:holomorphic_implicit_function_theorem}),
        there exist open neighborhoods \(U'\) of \(q\) such that \(f^{-1}(p)\cap U'\) is biholomorphic to an open subset of \(\bbC^{n-m}\).
    \end{proof}

    \begin{example}\label{eg:non_singular_complex_algebraic_variety_as_complex_manifold}
        Let \(X \subset \bbC^n\) be a complex algebraic variety defined by the vanishing of polynomials \(f_1, \ldots, f_m \in \bbC[z_1, \ldots, z_n]\).
        Suppose that \(X\) is non-singular, i.e., for every point \(p\in X\), the Jacobian matrix \(\left( \partial_{z_j}f_i(p) \right)_{i,j}\) has maximal rank \(r\).
        Then \(X\) is a complex submanifold of \(\bbC^n\) of complex dimension \(n - r\).
    \end{example}

    \begin{example}\label{eg:hypersurface_in_CPn_as_complex_manifold}
        A \emph{hypersurface} \(H\) in \(\bbC\bbP^n\) is the zero locus of a homogeneous polynomial \(f \in \bbC[z_0, z_1, \ldots, z_n]\).
        Suppose \(0\) is a regular value of \(f: \bbC^{n+1}\setminus \{0\} \to \bbC\).
        On each chart \(U_i \cong \bbC^n\) of \(\bbC\bbP^n\), it defines a holomorphic function \(f_i: U_i \to \bbC, [z] \mapsto z = (z_1, \ldots, z_{i-1}, 1, z_{i+1}, \ldots, z_n) \mapsto f(z)\).
        The regularity condition implies that \(0\) is a regular value of each \(f_i\).
        Hence \(H \cap U_i = f_i^{-1}(0)\) is a complex submanifold of \(U_i\) of complex dimension \(n-1\) by \cref{prop:complex_submanifold_by_holomorphic_implicit_function_theorem}.
        Gluing these local pieces together, we see that \(H\) is a complex submanifold of \(\bbC\bbP^n\) of complex dimension \(n-1\).
        % Concretely, the Fermat hypersurface \(H\), which is defined by the equation \(z_0^m + z_1^m + \ldots + z_n^m = 0\) in \(\bbC\bbP^n\) for some integer \(m \geq 1\), is a complex manifold of complex dimension \(n-1\).
    \end{example}

    \begin{proposition}\label{prop:quotient_by_a_discrete_group_of_holomorphic_automorphisms}
        Let \(M\) be a complex manifold and let \(G\) be a discrete group acting on \(M\) by holomorphic automorphisms.
        If the action is free and properly discontinuous, then the quotient space \(M/G\) is a complex manifold and the quotient map \(\pi: M\to M/G\) is a holomorphic covering map.
    \end{proposition}
    \begin{proof}
        For every point \(p\in M/G\), choose a point \(q\in M\) such that \(\pi(q) = p\).
        Since the action is free and properly discontinuous (see \cref{rmk:properly_discontinuous_action}), there exists an open neighborhood \(U\) of \(q\) such that \(gU \cap U = \emptyset\) for all \(g\in G\setminus \{e\}\).
        Then \(\pi|_U: U \to \pi(U)\) is a homeomorphism.
        This gives a chart of \(M/G\) around \(p\).
        If we have two such charts \((\pi(U),\varphi)\) and \((\pi(U'),\varphi')\) of \(M/G\) whose intersection is non-empty, WLOG, assume that \(U \cap U' \neq \emptyset\).
        Then \(\pi^{-1}(\pi(U) \cap \pi(U')) = \bigsqcup_{g\in G} g(U \cap U')\).
        The transition map of \(U\) and \(U'\) gives the transition map of \(\pi(U)\) and \(\pi(U')\).
        Since the action of \(G\) is by holomorphic automorphisms, the transition maps are holomorphic.
    \end{proof}
    \begin{remark}\label{rmk:properly_discontinuous_action}
        Recall that an action of a group \(G\) on a topological space \(X\) is said to be \emph{properly discontinuous} if for every compact subset \(K \subset X\), the set \(\{g\in G: gK \cap K \neq \emptyset\}\) is finite.
        If \(G\) is a discrete group acting on a manifold \(M\) by diffeomorphisms, 
        then the action is properly discontinuous and free if and only if for every point \(p\in M\), 
        there exists an open neighborhood \(U\) of \(p\) such that \(gU \cap U = \emptyset\) for all \(g\in G\setminus \{e\}\).
    \end{remark}

    \begin{example}\label{eg:elliptic_curves_as_complex_manifolds_by_quotient_of_C}
        Let \(\Lambda \subset \bbC\) be a lattice, i.e., a discrete subgroup of \(\bbC\) generated by two \(\bbR\)-linearly independent complex numbers.
        Then \(\Lambda\) is isomorphic to \(\bbZ^2\) as an abstract group and acts on \(\bbC\) by translations, which are holomorphic automorphisms of \(\bbC\).
        Then the quotient space \(\bbC/\Lambda\) is a complex manifold of complex dimension \(1\) by \cref{prop:quotient_by_a_discrete_group_of_holomorphic_automorphisms}.
        Such a complex manifold is called an \emph{elliptic curve}.
        As real manifolds, it is diffeomorphic to \(S^1 \times S^1\).
    \end{example}

    \begin{example}\label{eg:hopf_manifolds_as_complex_manifolds_by_quotient_of_Cn}
        Fix \(\alpha \in \bbC^\times\) with \(|\alpha| \neq 1\).
        Let \(\bbZ\) act on \(\bbC^n\setminus \{0\}\) by \(k \cdot z = \alpha^k z\) for every \(k\in \bbZ\) and \(z\in \bbC^n\setminus \{0\}\).
        This action is free and properly discontinuous.
        Then the quotient space \((\bbC^n\setminus \{0\})/\bbZ\) is a complex manifold of complex dimension \(n\) by \cref{prop:quotient_by_a_discrete_group_of_holomorphic_automorphisms}.
        Such a complex manifold is called a \emph{Hopf manifold}.
    \end{example}

    \begin{example}\label{eg:Iwasawa_manifold_as_complex_manifold_by_quotient_of_complex_Heisenberg_group}
        Let 
        \[ M = \left\{ \begin{pmatrix}
            1 & z_1 & z_3 \\
            0 & 1 & z_2 \\
            0 & 0 & 1
        \end{pmatrix} \;\middle|\; z_1, z_2, z_3 \in \bbC \right\} \]
        be the complex Heisenberg group, which is biholomorphic to \(\bbC^3\).
        Let \(\Gamma := M \cap \GL(3,\bbZ[\im])\).
        Then \(\Gamma\) is a discrete subgroup of \(M\) and acts on \(M\) by left multiplication, which are holomorphic automorphisms of \(M\).
        The action is free and properly discontinuous.
        Then the quotient space \(M/\Gamma\) is a complex manifold of complex dimension \(3\) by \cref{prop:quotient_by_a_discrete_group_of_holomorphic_automorphisms}.
        It is called the \emph{Iwasawa manifold}.
        One can replace \(\Gamma\) by other cocompact discrete subgroups of \(M\).
    \end{example}

\subsection{Almost Complex Structures}

    Let \(X\) be a complex manifold of complex dimension \(n\).
    The tangent bundle \(TX\) is a real vector bundle of rank \(2n\).
    There is a natural endomorphism \(J: TX\to TX\) induced by the complex structure of \(X\), i.e., for every point \(p\in X\), \(J_p: T_pX\to T_pX\) is the multiplication by \(\im\).
    We have \(J^2 = -\id\).

    \begin{definition}\label{def:almost_complex_structure}
        Let \(M\) be a smooth manifold of real dimension \(2n\).
        An \emph{almost complex structure} on \(M\) is a smooth endomorphism \(J: TM\to TM\) such that \(J^2 = -\id\).
        The pair \((M,J)\) is called an \emph{almost complex manifold}.
    \end{definition}

    % \begin{definition}\label{def:J_holomorphic_function}
    %     Let \((M,J)\) be an almost complex manifold.
    %     A smooth function \(f: M\to \bbC\) is called \emph{\(J\)-holomorphic} if
    %     \[ \upd f \circ J = \im \cdot \upd f. \]
    % \end{definition}

    \begin{question}\label{qs:existence_of_almost_complex_structure}
        Given a smooth manifold \(M\) of real dimension \(2n\), when does it admit an almost complex structure?
        Is such an almost complex structure unique?
    \end{question}

    Giving an almost complex structure \(J\) on a smooth manifold \(M\) is equivalent to giving the tangent bundle \(TM\) the structure of a complex vector bundle.
    Hence the existence of almost complex structures is a purely topological problem.
    Note that to find a complex structure on \(M\) needs to solve some non-linear partial differential equations, which is much harder.

    \begin{example}\label{eg:almost_complex_structure_on_S6}
        The \(6\)-sphere \(S^6\) admits an almost complex structure.
        In fact, \(S^6\) can be identified with the unit sphere in the imaginary octonions \(\Image \bbO\) (see \cref{rmk:fundamental_facts_about_octonion_for_almost_complex_structure_on_S6}).
        Denote by \(m(x,y)\) the octonionic multiplication of \(x,y\in \bbO\).
        For every point \(p\in S^6\), the tangent space \(T_pS^6\) can be identified with the orthogonal complement of \(\bbR p\) in \(\Image \bbO\).
        Define \(J_p: T_pS^6 \to T_pS^6\) by \(J_p(v) = m(p,v)\).
        Then \(J_p^2(v) = p(pv) = -v\) for every \(v\in T_pS^6\).
        Thus we get an almost complex structure on \(S^6\).
    \end{example}
    \begin{remark}\label{rmk:fundamental_facts_about_octonion_for_almost_complex_structure_on_S6}
        Recall some fundamental facts about the octonions \(\bbO\):
        \begin{enumerate}
            \item \(\bbO\) is an \(8\)-dimensional normed vector space over \(\bbR\) with an orthogonal basis \(\{1\}\cup \{e_i|i=1,\ldots,7\}\).
                The subspace spanned by \(\{e_i\}\) is called the space of imaginary octonions and denoted by \(\Image \bbO\).
            \item The multiplication \(m: \bbO \times \bbO \to \bbO\) is a bilinear map and satisfies the distributive law and the norm multiplicative law \(\|xy\| = \|x\|\|y\|\) for all \(x,y\in \bbO\).
                It is given by the following Fano plane \(\bbP^2(\bbF_2)\):
                \begin{center}
                \usetikzlibrary{calc, decorations.markings}
                \begin{tikzpicture}[scale=1.5, >=stealth,
                    midarrow/.style={decoration={markings, mark=at position 0.5 with {\arrow{>}}}, postaction={decorate}}]
                    % Define the vertices of the triangle
                    \coordinate (A) at (0, 2);              %e6
                    \coordinate (B) at (-1.732, -1);        %e3
                    \coordinate (C) at (1.732, -1);         %e5
                    % Define the center and midpoints
                    \coordinate (O) at (0, 0);              %e7
                    \coordinate (D) at (0, -1);             %e2
                    \coordinate (E) at (0.866, 0.5);        %e1
                    \coordinate (F) at (-0.866, 0.5);       %e4

                    % Draw all 7 lines of the Fano plane with arrows in middle
                    % Line 1: e3 -> e4 -> e6
                    \draw[thick, midarrow] (B) -- (F);
                    \draw[thick, midarrow] (F) -- (A);
                    % Line 2: e6 -> e1 -> e5
                    \draw[thick, midarrow] (A) -- (E);
                    \draw[thick, midarrow] (E) -- (C);
                    % Line 3: e5 -> e2 -> e3
                    \draw[thick, midarrow] (C) -- (D);
                    \draw[thick, midarrow] (D) -- (B);
                    % Line 4: e6 -> e7 -> e2
                    \draw[thick, midarrow] (A) -- (O);
                    \draw[thick, midarrow] (O) -- (D);
                    % Line 5: e5 -> e7 -> e4
                    \draw[thick, midarrow] (C) -- (O);
                    \draw[thick, midarrow] (O) -- (F);
                    % Line 6: e3 -> e7 -> e1
                    \draw[thick, midarrow] (B) -- (O);
                    \draw[thick, midarrow] (O) -- (E);
                    % Line 7: e1, e2, e7 (as a circle passing through these three points)
                    \draw[thick, decoration={markings, mark=at position 0.166 with {\arrow{<}}, mark=at position 0.5 with {\arrow{<}}, mark=at position 0.833 with {\arrow{<}}}, postaction={decorate}] 
                                            (0, 0) circle (1);                               
                    % Place the labels
                    \node at (A) [above]        {\(e_6\)};
                    \node at (B) [below left]   {\(e_3\)};
                    \node at (C) [below right]  {\(e_5\)};
                    \node at (D) [below]        {\(e_7\)};
                    \node at (E) [right]        {\(e_2\)};
                    \node at (F) [left]         {\(e_1\)};
                    \node at (O) [right]        {\(e_4\)};
                    
                    % Add dots for the points
                    \filldraw[black] (A) circle (2pt);
                    \filldraw[black] (B) circle (2pt);
                    \filldraw[black] (C) circle (2pt);
                    \filldraw[black] (D) circle (2pt);
                    \filldraw[black] (E) circle (2pt);
                    \filldraw[black] (F) circle (2pt);
                    \filldraw[black] (O) circle (2pt);
                \end{tikzpicture}
                \end{center}
                If \(e_i \to e_j \to e_k\) is a directed line in the Fano plane, then \(e_i e_j = e_k\), \(e_j e_k = e_i\), and \(e_k e_i = e_j\).
                The multiplication is anti-commutative, i.e., \(e_i e_j = - e_j e_i\) for all \(i\neq j\).
                And we have \(e_i^2 = -1\) for all \(i\).
        \end{enumerate}
        \Yang{To be checked...}
    \end{remark}

    Let \((M,J)\) be an almost complex manifold.
    Then the complexified tangent bundle \(TM_\bbC  := TM \otimes_\bbR \bbC\) splits into the direct sum of two complex subbundles
    \[ T M_\bbC = T^{1,0}M \oplus T^{0,1}M, \]
    where 
    \[ T^{1,0}M := \ker (\im \id - J), \quad T^{0,1}M := \ker (\im \id + J). \]
    We have \(\overline{T^{1,0}M} = T^{0,1}M\) and both \(T^{1,0}M\) and \(T^{0,1}M\) are complex vector bundles of rank \(n\).
    This decomposition induces a decomposition of the complexified cotangent bundle
    \[ \Omega^1(M) := (TM_\bbC)^* = (T^{1,0}M)^* \oplus (T^{0,1}M)^* =: \Omega^{1,0}(M) \oplus \Omega^{0,1}(M). \]
    More generally, for every \(p,q \geq 0\), define
    \[ \Omega^{p,q}(M) := \wedge^p (T^{1,0}M)^* \otimes \wedge^q (T^{0,1}M)^* \subset \wedge^{p+q} \Omega^1(M). \]
    Then we have the decomposition
    \[ \Omega^k(M) \coloneqq \wedge^k \Omega^1(M) = \bigoplus_{p+q=k} \Omega^{p,q}(M). \]
    The elements of \(\Omega^{p,q}(M)\) are called \emph{differential forms of type \((p,q)\)} or \emph{\((p,q)\)-forms} for short.
    % \Yang{To be continued...}

    Recall the \emph{exterior differential operator} \(\upd:\Omega^k(M) \to \Omega^{k+1}(M)\) is locally given by 
    \[ \upd\left(\sum_I f_I \upd x_I\right) = \sum_I \sum_{j=1}^{2n} \frac{\partial f_I}{\partial x_j} \upd x_j \wedge \upd x_I, \]
    where \(I\) runs over all multi-indices with \(|I| = k\) and \(x_1, \ldots, x_{2n}\) are local coordinates on \(M\).

    \begin{proposition}\label{prop:decomposition_of_differential_operator_on_almost_complex_manifold}
        There exist differential operators
        \[ \partial: \Omega^{p,q}(M) \to \Omega^{p+1,q}(M), \quad \mu: \Omega^{p,q}(M) \to \Omega^{p+2,q-1}(M) \]
        such that
        \[ \upd = \partial + \overline{\partial} + \mu + \overline{\mu}. \]
    \end{proposition}
    In a diagram:
    \begin{center}
        \begin{tikzpicture}
            \matrix (m) [matrix of math nodes, column sep=2em, row sep=2em] {
                \Omega^{p-1,q+2} & \bullet & \bullet & \bullet & \bullet \\
                \bullet & \Omega^{p,q+1}  & \bullet & \bullet & \bullet \\
                \bullet & \Omega^{p,q} & \Omega^{p+1,q} & \bullet & \bullet \\
                \bullet & \bullet & \bullet & \Omega^{p+2,q-1} & \bullet \\
                    &  &  & \Omega^{k} & \Omega^{k+1} \\
                };
                % Draw a horizontal line before the row containing \Omega^k (above row 5)
                \draw (m-4-1.south) -- (m-4-5.south);
                % Draw a dashed line passing through \Omega^{p-1,q+1} and \Omega^k
                \draw[dashed] (m-2-1) -- (m-5-4);
                \draw[dashed] (m-1-1) -- (m-5-5);
                % Draw arrows for the operators
                \draw[->] (m-3-2) -- (m-3-3) node[midway, above] {\(\partial\)};
                \draw[->] (m-3-2) -- (m-2-2) node[midway, right] {\(\overline{\partial}\)};
                \draw[->] (m-3-2) -- (m-4-4) node[midway, right] {\(\mu\)};
                \draw[->] (m-3-2) -- (m-1-1) node[midway, above] {\(\overline{\mu}\)};
                \draw[->] (m-5-4) -- (m-5-5) node[midway, above] {\(\upd\)};
        \end{tikzpicture} 
    \end{center}
    \begin{proof}[Proof of \cref{prop:decomposition_of_differential_operator_on_almost_complex_manifold}]

        \Yang{To be continued...}
    \end{proof}

    % \begin{proposition}\label{prop:properties_of_differential_operators}
    %     The operators \(\partial\) and \(\mu\) satisfy the following properties:
    %     \begin{enumerate}
    %         \item 
    %     \end{enumerate}
    %     \Yang{To be continued...}
    % \end{proposition}

    \begin{definition}\label{def:Nijenhuis_operator}
        The operator \(\mu\) in \cref{prop:decomposition_of_differential_operator_on_almost_complex_manifold} is called the \emph{Nijenhuis operator} of the almost complex structure \(J\).
        If \(\mu = 0\), then \(J\) is called \emph{integrable}.
        In this case, we have \(\upd = \partial + \overline{\partial}\).
        % \Yang{To be continued...}
    \end{definition}

    \begin{example}\label{eg:Nijenhuis_operator_on_S6_for_the_almost_complex_structure_on_S6_by_octonion}
        Let \(J\) be the almost complex structure on \(S^6\) defined in \cref{eg:almost_complex_structure_on_S6}.
        
        \Yang{To be continued...}
    \end{example}

    \begin{proposition}\label{prop:Nijenhuis_operator_on_complex_manifold}
        Let \((M,J)\) be an almost complex manifold.
        If \(J\) is induced by a complex structure on \(M\), then \(J\) is integrable, i.e., the Nijenhuis operator \(\mu = 0\).
    \end{proposition}
    \begin{proof}
        \Yang{To be continued...}
    \end{proof}

    The converse of \cref{prop:Nijenhuis_operator_on_complex_manifold} is also true, which is the famous Newlander-Nirenberg theorem.
    \Yang{To add reference...}

    \begin{theorem}[Newlander-Nirenberg Theorem]\label{thm:newlander_nirenberg_theorem}
        Let \((M,J)\) be an almost complex manifold of real dimension \(2n\).
        If \(\mu = 0\), then \(J\) is induced by a complex structure on \(M\).
    \end{theorem}

    \begin{proposition}\label{prop:integrable_almost_complex_structure_and_partial_square_zero}
        Let \((M,J)\) be an almost complex manifold.
        Then \(J\) is integrable if and only if \(\partial^2 = 0\).
    \end{proposition}


\subsection{Cohomology in complex manifolds}

    % Let \(M\) be a complex manifold.
    % Denote by \(\Omega_{\sm}^k(M)\) the space of smooth differential \(k\)-forms on \(M\) and by \(\Omega_{\sm}^{p,q}(M)\) the space of smooth \((p,q)\)-forms on \(M\).
    % Then \(\Omega_{\sm}^k(M) = \bigoplus_{p+q=k} \Omega_{\sm}^{p,q}(M)\).
    % Denote by \(\Omega_{\hol}^k(M)\) the space of holomorphic differential \(k\)-forms on \(M\).
    % Then we have \(\Omega_{\sm}^{k,0}(M) = \Omega_{\hol}^k(M) \ten_{\calO_M^{\hol}} \calO_M^{\sm}\).

    % There are several cohomology theories for complex manifolds.

    % \begin{definition}\label{def:de_rham_cohomology_of_complex_manifold}
    %     Let \(M\) be a complex manifold.
    %     The \emph{de Rham cohomology} of \(M\) is defined to be the de Rham cohomology of the underlying smooth manifold of \(M\):
    %     \[ H^k_{\dR}(M) := \frac{\Ker(\upd: \Omega^k(M) \to \Omega^{k+1}(M))}{\Image(\upd: \Omega^{k-1}(M) \to \Omega^k(M))}. \]
    % \end{definition}

    % \begin{definition}\label{def:dolbeault_cohomology_of_complex_manifold}
    %     Let \(M\) be a complex manifold.
    %     The \emph{Dolbeault cohomology} of \(M\) is defined to be
    %     \[ H^{p,q}_{\overline{\partial}}(M) := \frac{\Ker(\overline{\partial}: \Omega^{p,q}(M) \to \Omega^{p,q+1}(M))}{\Image(\overline{\partial}: \Omega^{p,q-1}(M) \to \Omega^{p,q}(M))}. \]
    % \end{definition}

    % \begin{definition}\label{def:Bott_Chern_cohomology_of_complex_manifold}
    %     Let \(M\) be a complex manifold.
    %     The \emph{Bott-Chern cohomology} of \(M\) is defined to be
    %     \[ H^{p,q}_{\mathrm{BC}}(M) := \frac{\Ker(\upd: \Omega^{p,q}(M) \to \Omega^{p+1,q}(M) \oplus \Omega^{p,q+1}(M))}{\Image(\partial\overline{\partial}: \Omega^{p-1,q-1}(M) \to \Omega^{p,q}(M))}. \]
    %     \Yang{To be checked...}
    % \end{definition}

    % \begin{definition}\label{def:Aeppli_cohomology_of_complex_manifold}
    %     Let \(M\) be a complex manifold.
    %     The \emph{Aeppli cohomology} of \(M\) is defined to be
    %     \[ H^{p,q}_{\mathrm{A}}(M) := \frac{\Ker(\partial\overline{\partial}: \Omega^{p,q}(M) \to \Omega^{p+1,q+1}(M))}{\Image(\partial: \Omega^{p-1,q}(M) \to \Omega^{p,q}(M)) + \Image(\overline{\partial}: \Omega^{p,q-1}(M) \to \Omega^{p,q}(M))}. \]
    %     \Yang{To be checked...}
    % \end{definition}

    % There are natural maps between these cohomology theories.
    % \Yang{To be continued...}

    % \begin{proposition}\label{prop:dolbeault_cohomology_of_polydisc}
    %     Let \(\Delta^n = \{(z_1, \ldots, z_n) \in \bbC^n: |z_i| < 1, i=1,\ldots,n\}\) be the unit polydisc in \(\bbC^n\).
    %     Then
    %     \[ H^{p,q}_{\overline{\partial}}(\Delta^n) = \begin{cases}
    %         \Omega_{\hol}^p(\Delta^n), & q=0, \\
    %         0, & q > 0.
    %     \end{cases} \]
    %     \Yang{To be checked...}
    % \end{proposition}