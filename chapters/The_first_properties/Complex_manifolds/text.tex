\section{Complex Manifolds}

\subsection{Definition and Examples}

    \begin{definition}\label{def:complex_manifold}
        A \emph{complex manifold} of complex dimension \(n\) is a topological space \(M\) such that
        \begin{enumerate}
            \item \(M\) is Hausdorff and second countable;
            \item \(M\) is locally homeomorphic to \(\bbC^n\), i.e., for every point \(p\in M\), there exists an open neighborhood \(U\) of \(p\) and a homeomorphism \(\varphi: U\to V\subset \bbC^n\), where \(V\) is an open subset of \(\bbC^n\),
                The pair \((U,\varphi)\) is called a \emph{chart};
            \item if \((U,\varphi)\) and \((U',\varphi')\) are two charts with \(U\cap U'\neq \emptyset\), then the transition map
                \[
                    \varphi'\circ \varphi^{-1}: \varphi(U\cap U')\to \varphi'(U\cap U')
                \]
                is holomorphic.
        \end{enumerate}
        The collection of all charts \(\{(U_\alpha,\varphi_\alpha)\}\) that cover \(M\) is called an \emph{atlas}.
        If the atlas is maximal, it is called a \emph{complex structure} on \(M\).
    \end{definition}

    Another way to define complex manifolds is to use the language of ringed spaces.

    \begin{definition}\label{def:complex_manifold_as_ringed_space}
        A \emph{complex manifold} of complex dimension \(n\) is a locally ringed space \((M,\calO_M)\) such that
        \begin{enumerate}
            \item \(M\) is Hausdorff and second countable;
            \item for every point \(p\in M\), there exists an open neighborhood \(U\) of \(p\) such that \((U,\calO_M|_U)\) is isomorphic to \((B,\calO_B)\), where \(B\) is the unit open ball in \(\bbC^n\) and \(\calO_B\) is the sheaf of holomorphic functions on \(B\).
        \end{enumerate}
    \end{definition}

    \begin{question}\label{qs:existence_and_uniqueness_of_complex_structure}
        Given a topological space \(M\) that is Hausdorff and second countable, when does it admit a complex structure?
        Is such a complex structure unique?
    \end{question}

    For complex dimension \(1\), the answer is positive and well-known.
    For higher dimensions, the answer is negative in general.
    In particular, does the \(6\)-sphere \(S^6\) admit a complex structure?
    This is a famous open problem in complex geometry.

    \begin{question}\label{qs:complex_structure_on_S6}
        Does the \(6\)-sphere \(S^6\) admit a complex structure?
    \end{question}

    \begin{definition}\label{def:holomorphic_map_between_complex_manifolds}
        Let \(M\) and \(N\) be two complex manifolds.
        A continuous map \(f: M\to N\) is called \emph{holomorphic} if for every point \(p\in M\), there exist charts \((U,\varphi)\) of \(M\) around \(p\) and \((V,\psi)\) of \(N\) around \(f(p)\) such that
        \[
            \psi\circ f\circ \varphi^{-1}: \varphi(U\cap f^{-1}(V))\to \psi(f(U)\cap V)
        \]
        is holomorphic.
        \Yang{To be continued...}
    \end{definition}

    \begin{definition}\label{def:submanifold_of_complex_manifold}
        Let \(M\) be a complex manifold of complex dimension \(n\).
        A subset \(S\subset M\) is called a \emph{complex submanifold} of complex dimension \(k\) if for every point \(p\in S\), there exist a chart \((U,\varphi)\) of \(M\) around \(p\) such that
        \[
            \varphi(U\cap S) = \varphi(U)\cap (\bbC^k\times \{0\}) \subset \bbC^n,
        \]
        where we identify \(\bbC^n\) with \(\bbC^k\times \bbC^{n-k}\).
        \Yang{To be continued...}
    \end{definition}

    \begin{example}\label{eg:complex_vector_space_as_complex_manifold}
        Any complex vector space \(V\) of complex dimension \(n\) is a complex manifold of complex dimension \(n\).        
    \end{example}

    \begin{example}\label{eg:complex_manifold_by_holomorphic_implicit_function_theorem}
        Let \(f: \bbC^{n+m}\to \bbC^m\) be a holomorphic function.
        Suppose that \(0\) is a regular value of \(f\), i.e., the Jacobian matrix \(Df_p\) is surjective for every \(p\in f^{-1}(0)\).
        Then by the holomorphic implicit function theorem (\cref{thm:holomorphic_implicit_function_theorem}),
        for every point \(p\in f^{-1}(0)\), there exist open neighborhoods \(U\) of \(p\) and \(V\) of \(0\) such that \(f^{-1}(0)\cap U\) is biholomorphic to an open subset of \(\bbC^n\).
        Thus \(f^{-1}(0)\) is a complex manifold of complex dimension \(n\).
        % the preimage \(f^{-1}(0)\) is a complex manifold of complex dimension \(n\).
        In particular, any non-singular complex algebraic variety is a complex manifold.
        \Yang{To be continued...}
    \end{example}

    \begin{example}\label{eg:complex_projective_space_as_complex_manifold}
        The complex projective space \(\bbC\bbP^n:= \bbC^{n+1}\setminus \{0\}/\bbC^\times\) is a complex manifold of complex dimension \(n\).
        In fact, \(\bbC\bbP^n\) can be covered by \(n+1\) charts, each of which is biholomorphic to \(\bbC^n\).
        For example, the chart \(U_0 = \{[z_0: z_1: \cdots: z_n]\in \bbC\bbP^n: z_0\neq 0\}\) is biholomorphic to \(\bbC^n\) via the map
        \[
            [z_0: z_1: \cdots: z_n] \mapsto \left(\frac{z_1}{z_0}, \frac{z_2}{z_0}, \ldots, \frac{z_n}{z_0}\right).
        \]
        The other charts are defined similarly.
    \end{example}

    \begin{proposition}\label{prop:zero_locus_of_homogeneous_polynomial_in_CPn}
        Let \(f:\bbC^{n+1}\setminus \{0\} \to \bbC\) be a homogeneous polynomial of degree \(d\) such that \(0\) is a regular value of \(f\).
        Then \(f^{-1}(0)/\bbC^\times\) is a complex submanifold of \(\bbC\bbP^n\) of complex dimension \(n-1\).
    \end{proposition}
    \begin{proof}
        \Yang{To be continued...}
    \end{proof}

    \begin{example}\label{eg:hopf_manifolds}
        The Hopf manifolds are a class of complex manifolds that can be constructed as quotients of complex vector spaces by the action of a group of automorphisms.
        \Yang{To be continued...}
    \end{example}

    \begin{example}\label{eg:Iwasawa_manifold}
        The Iwasawa manifold is a complex manifold that can be constructed as a quotient of the complex Heisenberg group by a discrete subgroup.
        \Yang{To be continued...}
    \end{example}

\subsection{Almost Complex Structures}

    Let \(X\) be a complex manifold of complex dimension \(n\).
    The tangent bundle \(TX\) is a real vector bundle of rank \(2n\).
    There is a natural endomorphism \(J: TX\to TX\) induced by the complex structure of \(X\), i.e., for every point \(p\in X\), \(J_p: T_pX\to T_pX\) is the multiplication by \(\im\).
    We have \(J^2 = -\id\).

    \begin{definition}\label{def:almost_complex_structure}
        Let \(M\) be a smooth manifold of real dimension \(2n\).
        An \emph{almost complex structure} on \(M\) is a smooth endomorphism \(J: TM\to TM\) such that \(J^2 = -\id\).
        The pair \((M,J)\) is called an \emph{almost complex manifold}.
    \end{definition}

    \begin{definition}\label{def:J_holomorphic_function}
        Let \((M,J)\) be an almost complex manifold.
        A smooth function \(f: M\to \bbC\) is called \emph{\(J\)-holomorphic} if
        \[ \upd f \circ J = \im \cdot \upd f. \]
    \end{definition}

    \begin{question}\label{qs:existence_of_almost_complex_structure}
        Given a smooth manifold \(M\) of real dimension \(2n\), when does it admit an almost complex structure?
        Is such an almost complex structure unique?
    \end{question}

    Giving an almost complex structure \(J\) on a smooth manifold \(M\) is equivalent to giving the tangent bundle \(TM\) the structure of a complex vector bundle.
    Hence the existence of almost complex structures is a purely topological problem.
    Note that to find a complex structure on \(M\) needs to solve some non-linear partial differential equations, which is much harder.

    \begin{example}\label{eg:almost_complex_structure_on_S6}
        The \(6\)-sphere \(S^6\) admits an almost complex structure.
        In fact, \(S^6\) can be identified with the unit sphere in the imaginary octonions \(\Image \bbO\).
        Denote by \(m(x,y)\) the octonionic multiplication of \(x,y\in \bbO\).
        For every point \(p\in S^6\), the tangent space \(T_pS^6\) can be identified with the orthogonal complement of \(\bbR p\) in \(\Im \bbO\).
        Define \(J_p: T_pS^6 \to T_pS^6\) by \(J_p(v) = m(p,v)\).
        Then \(J_p^2(v) = p(pv) = -v\) for every \(v\in T_pS^6\).
        Thus we get an almost complex structure on \(S^6\).
        \Yang{To be continued...}
    \end{example}
    \begin{remark}\label{rmk:fundamental_facts_about_octonion_for_almost_complex_structure_on_S6}
        Recall some fundamental facts about the octonions \(\bbO\):

        \Yang{To be continued...}
    \end{remark}

% \subsection{The exterior differential and the Nijenhuis tensor}

    Let \((M,J)\) be an almost complex manifold.
    Then the complexified tangent bundle \(TM_\bbC  := TM \otimes_\bbR \bbC\) splits into the direct sum of two complex subbundles
    \[ T M_\bbC = T^{1,0}M \oplus T^{0,1}M, \]
    where 
    \[ T^{1,0}M := \ker (\im \id - J), \quad T^{0,1}M := \ker (\im \id + J). \]
    We have \(\overline{T^{1,0}M} = T^{0,1}M\) and both \(T^{1,0}M\) and \(T^{0,1}M\) are complex vector bundles of rank \(n\).
    This decomposition induces a decomposition of the complexified cotangent bundle
    \Yang{To be continued...}

    \begin{proposition}\label{prop:decomposition_of_differential_operator}
        There exist differential operators
        \[ \P: \Omega^{p,q}(M) \to \Omega^{p+1,q}(M), \quad \mu: \Omega^{p,q}(M) \to \Omega^{p+2,q-1}(M) \]
        such that
        \[ \upd = \P + \overline{\P} + \mu + \overline{\mu}. \]
    \end{proposition}

    In a diagram:


    \begin{proof}[Proof of \cref{prop:decomposition_of_differential_operator}]
        \Yang{To be continued...}
    \end{proof}

    \begin{proposition}\label{prop:properties_of_differential_operators}
        The operators \(\P\) and \(\mu\) satisfy the following properties:
        \begin{enumerate}
            \item 
        \end{enumerate}
        \Yang{To be continued...}
    \end{proposition}