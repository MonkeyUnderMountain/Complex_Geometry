\section{Cohomology Theories in Complex Geometry}


\subsection{Differential forms}

    Let \(M\) be a complex manifold.
    
    Denote by \(\Omega_{\sm}^k(M)\) the space of smooth differential \(k\)-forms on \(M\) and by \(\Omega_{\sm}^{p,q}(M)\) the space of smooth \((p,q)\)-forms on \(M\).
    Then \(\Omega_{\sm}^k(M) = \bigoplus_{p+q=k} \Omega_{\sm}^{p,q}(M)\).
    Denote by \(\Omega_{\hol}^k(M)\) the space of holomorphic differential \(k\)-forms on \(M\).
    Then we have \(\Omega_{\sm}^{k,0}(M) = \Omega_{\hol}^k(M) \ten_{\calO_M^{\hol}} \calO_M^{\sm}\).

    Recall that we have the exterior derivative
    \[ \upd: \Omega_{\sm}^k(M) \to \Omega_{\sm}^{k+1}(M), \]
    which can be decomposed as
    \[ \upd = \partial + \overline{\partial}, \]
    where 
    \[ \partial: \Omega_{\sm}^{p,q}(M) \to \Omega_{\sm}^{p+1,q}(M), \quad \overline{\partial}: \Omega_{\sm}^{p,q}(M) \to \Omega_{\sm}^{p,q+1}(M). \]
    In a diagram, we have:
    \begin{center}
        \begin{tikzpicture}
            \matrix (m) [matrix of math nodes, column sep=2em, row sep=2em] {
                \bullet & \bullet & \bullet & \bullet & \bullet \\
                \bullet & \Omega^{p,q+1}  & \bullet & \bullet & \bullet \\
                \bullet & \Omega^{p,q} & \Omega^{p+1,q} & \bullet & \bullet \\
                \bullet & \bullet & \bullet & \bullet & \bullet \\
                    &  &  & \Omega^{k} & \Omega^{k+1} \\
                };
                % Draw a horizontal line before the row containing \Omega^k (above row 5)
                \draw (m-4-1.south) -- (m-4-5.south);
                % Draw a dashed line passing through \Omega^{p-1,q+1} and \Omega^k
                \draw[dashed] (m-2-1) -- (m-5-4);
                \draw[dashed] (m-1-1) -- (m-5-5);
                % Draw arrows for the operators
                \draw[->] (m-3-2) -- (m-3-3) node[midway, above] {\(\partial\)};
                \draw[->] (m-3-2) -- (m-2-2) node[midway, right] {\(\overline{\partial}\)};
                % \draw[->] (m-3-2) -- (m-4-4) node[midway, right] {\(\mu\)};
                % \draw[->] (m-3-2) -- (m-1-1) node[midway, above] {\(\overline{\mu}\)};
                \draw[->] (m-5-4) -- (m-5-5) node[midway, above] {\(\upd\)};
        \end{tikzpicture} 
    \end{center}


\subsection{Various cohomology theories}

    % Let \(M\) be a complex manifold.
    % Denote by \(\Omega_{\sm}^k(M)\) the space of smooth differential \(k\)-forms on \(M\) and by \(\Omega_{\sm}^{p,q}(M)\) the space of smooth \((p,q)\)-forms on \(M\).
    % Then \(\Omega_{\sm}^k(M) = \bigoplus_{p+q=k} \Omega_{\sm}^{p,q}(M)\).
    % Denote by \(\Omega_{\hol}^k(M)\) the space of holomorphic differential \(k\)-forms on \(M\).
    % Then we have \(\Omega_{\sm}^{k,0}(M) = \Omega_{\hol}^k(M) \ten_{\calO_M^{\hol}} \calO_M^{\sm}\).

    There are several cohomology theories for complex manifolds.

    \begin{definition}\label{def:singular_cohomology_on_complex_manifolds}
        Let \(M\) be a complex manifold.
        The \emph{singular cohomology} of \(M\) with coefficients in a ring \(R\) is defined to be the singular cohomology of the underlying topological space \(|M|\) of \(M\):
        \[ H^k_{\sing}(M; R) := H^k_{\sing}(|M|; R). \]
    \end{definition}

    \begin{definition}\label{def:de_rham_cohomology_of_complex_manifold}
        Let \(M\) be a complex manifold.
        The \emph{de Rham cohomology} of \(M\) is defined to be the de Rham cohomology of the underlying smooth manifold of \(M\):
        \[ H^k_{\dR}(M) := \frac{\Ker(\upd: \Omega^k(M) \to \Omega^{k+1}(M))}{\Image(\upd: \Omega^{k-1}(M) \to \Omega^k(M))}. \]
    \end{definition}

    \Yang{Smooth section or holomorphic section?}

    \begin{definition}\label{def:dolbeault_cohomology_of_complex_manifold}
        Let \(M\) be a complex manifold.
        The \emph{Dolbeault cohomology} of \(M\) is defined to be
        \[ H^{p,q}_{\overline{\partial}}(M) := \frac{\Ker(\overline{\partial}: \Omega^{p,q}(M) \to \Omega^{p,q+1}(M))}{\Image(\overline{\partial}: \Omega^{p,q-1}(M) \to \Omega^{p,q}(M))}. \]
    \end{definition}

    % \begin{definition}\label{def:Bott_Chern_cohomology_of_complex_manifold}
    %     Let \(M\) be a complex manifold.
    %     The \emph{Bott-Chern cohomology} of \(M\) is defined to be
    %     \[ H^{p,q}_{\mathrm{BC}}(M) := \frac{\Ker(\upd: \Omega^{p,q}(M) \to \Omega^{p+1,q}(M) \oplus \Omega^{p,q+1}(M))}{\Image(\partial\overline{\partial}: \Omega^{p-1,q-1}(M) \to \Omega^{p,q}(M))}. \]
    %     \Yang{To be checked...}
    % \end{definition}

    % \begin{definition}\label{def:Aeppli_cohomology_of_complex_manifold}
    %     Let \(M\) be a complex manifold.
    %     The \emph{Aeppli cohomology} of \(M\) is defined to be
    %     \[ H^{p,q}_{\mathrm{A}}(M) := \frac{\Ker(\partial\overline{\partial}: \Omega^{p,q}(M) \to \Omega^{p+1,q+1}(M))}{\Image(\partial: \Omega^{p-1,q}(M) \to \Omega^{p,q}(M)) + \Image(\overline{\partial}: \Omega^{p,q-1}(M) \to \Omega^{p,q}(M))}. \]
    %     \Yang{To be checked...}
    % \end{definition}

    % There are natural maps between these cohomology theories.
    % \Yang{To be continued...}

    \begin{proposition}\label{prop:dolbeault_cohomology_of_polydisc}
        Let \(\Delta^n = \{(z_1, \ldots, z_n) \in \bbC^n: |z_i| < 1, i=1,\ldots,n\}\) be the unit polydisc in \(\bbC^n\).
        Then
        \[ H^{p,q}_{\overline{\partial}}(\Delta^n) = \begin{cases}
            \Omega_{\hol}^p(\Delta^n), & q=0, \\
            0, & q > 0.
        \end{cases} \]
        \Yang{To be checked...}
    \end{proposition}