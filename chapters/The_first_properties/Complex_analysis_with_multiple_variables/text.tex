\section{Complex analysis with multiple variables}

In this section, we introduce some basic concepts and results in complex analysis with multiple variables.

\subsection{Holomorphic functions}

    We identify \(\bbC^n \cong \bbR^{2n}\).

    \begin{definition}\label{def:differentiable_and_holomorphic_map}
        A continuous map \(f: \bbR^{2n} \to \bbR^{2m}\) is \emph{differentiable} at \(p \in \bbR^{2n}\) if there exists a linear map \(Df_p: \bbR^{2n} \to \bbR^{2m}\) such that 
        \[ f(z) = f(p) + Df_p(z - p) + o(|z - p|). \]

        A continuous map \(f: \bbC^n \to \bbC^m\) is \emph{holomorphic} at \(p \in \bbC^n\) if it is differentiable at \(p\) and \(Df_p\) is \(\bbC\)-linear, i.e., \(Df_p(\im z) = \im Df_p(z)\) for all \(z \in \bbC^n\).
    \end{definition}

    By a ``function'', we always mean a complex-valued function, i.e., a map \(f: \bbC^n \to \bbC\).
    Fix a coordinate system \(z = (z_1, \ldots, z_n)\) on \(\bbC^n\) and write \(z_j = x_j + iy_j\) for \(j = 1, \ldots, n\).
    Then a differentiable function \(f = u+iv: \bbC^n \to \bbC\) is holomorphic at \(p\) if and only if the Cauchy-Riemann equations hold:
    \[ \frac{\P u}{\P x_i}(p) = \frac{\P v}{\P y_i}(p), \quad \frac{\P u}{\P y_i}(p) = -\frac{\P v}{\P x_i}(p), \quad i = 1, \ldots, n. \]
    
    For convenience, we consider the complexified tangent space \(T\bbR^{2n} \ten_\bbR \bbC\) and introduce the following operators.

    \begin{definition}\label{def:Wirtinger_operators}
        The \emph{Wirtinger operators} are defined as 
        \[ \frac{\P}{\P z_j} := \frac{1}{2} \left( \frac{\P}{\P x_j} - \im \frac{\P}{\P y_j} \right), \quad \frac{\P}{\P \bar{z}_j} := \frac{1}{2} \left( \frac{\P}{\P x_j} + \im \frac{\P}{\P y_j} \right), \quad j = 1, \ldots, n. \]
    \end{definition}

    Then we can rewrite the Cauchy-Riemann equations as
    \[ \frac{\P f}{\P \bar{z}_j} = 0, \quad j = 1, \ldots, n. \]

    We summarize some properties of Wirtinger operators in the following proposition.
    \begin{proposition}\label{prop:properties_of_Wirtinger_operators}
        The Wirtinger operators satisfy the following properties:
        \begin{enumerate}
            \item \(\P_{z_j} z_i = \delta_{ij}\), \(\P_{z_j} \bar{z}_i = 0\), \(\P_{z_j} \bar{z}_i = 0\), \(\P_{z_j} \bar{z}_j = \delta_{ij}\);
            \item \(\overline{\left( \P_{z_j} f \right)} = \P_{\bar{z}_j} \bar{f}\);
            \item suppose we have \(\bbC^n \xrightarrow{g} \bbC^m \xrightarrow{f} \bbC^l\) and the coordinate on \(\bbC^m\) is \(w = (w_1, \ldots, w_m)\), then the chain rule holds:
                \begin{align*}
                    \frac{\P (f \circ g)}{\P z_j} &= \sum_{k=1}^m \frac{\P f}{\P w_k}(g(z)) \frac{\P g_k}{\P z_j}(z) + \sum_{k=1}^m \frac{\P f}{\P \bar{w}_k}(g(z)) \frac{\P \bar{g}_k}{\P z_j}(z), \\
                    \frac{\P (f \circ g)}{\P \bar{z}_j} &= \sum_{k=1}^m \frac{\P f}{\P w_k}(g(z)) \frac{\P g_k}{\P \bar{z}_j}(z) + \sum_{k=1}^m \frac{\P f}{\P \bar{w}_k}(g(z)) \frac{\P \bar{g}_k}{\P \bar{z}_j}(z).
                \end{align*}
        \end{enumerate}
    \end{proposition}
    \begin{proof}
        \Yang{By computation.}
    \end{proof}

    We can also consider the complexified of derivatives
    \[ Df_p^\bbC: T\bbR^{2n} \otimes_\bbR \bbC \to T\bbR^{2m} \otimes_\bbR \bbC. \]
    If we take \(\{\P_{z_i}, \P_{\bar{z}_i}\}_{i=1}^n\) as a basis of \(T^*\bbR^{2n} \otimes_\bbR \bbC\) and \(\{\P_{w_j}, \P_{\bar{w}_j}\}_{j=1}^m\) as a basis of \(T^*\bbR^{2m} \otimes_\bbR \bbC\), 
    then the matrix representation of \(Df_p^\bbC\) is
    \[ Df_p^\bbC = \begin{pmatrix}
        \frac{\P f}{\P z}(p) & \frac{\P f}{\P \bar{z}}(p) \\
        \overline{\frac{\P f}{\P \bar{z}}(p)} & \overline{\frac{\P f}{\P z}(p)}
    \end{pmatrix}. \]
    \Yang{To be checked}
    In particular, if \(f\) is holomorphic, then we have \(\det Df_p^\bbC = |\det(\P_z f)(p)|^2 \geq 0\).

    % Consider the complexified differential \(T\bbR^{2n} \otimes_\bbR \bbC\). 
    % We can extend the Wirtinger operators to complexified tangent vectors by linearity. 
    % Then \(\P_{z_j}\) and \(\P{\bar{z}_j}\) form a basis of \(T^*\bbR^{2n} \otimes_\bbR \bbC\).
    % If \(f\) is holomorphic, under this basis, its differential \(df\) can be written as
    % \[ df = \begin{pmatrix}
    %     \frac{\P f}{\P z} & \\
    %     & \overline{\frac{\P f}{\P z}}
    % \end{pmatrix} \].

    \begin{definition}\label{def:biholomorphic_map}
        A map \(f: \Omega \to \Omega'\) between two open sets \(\Omega \subset \bbC^n\) and \(\Omega' \subset \bbC^m\) is \emph{biholomorphic} if it is a bijection and both \(f\) and \(f^{-1}\) are holomorphic.
    \end{definition}

    If \(f\) is biholomorphic at \(p\), then \(m = n\) and \(\det Df_p > 0\).

    \begin{theorem}[Holomorphic inverse function theorem]\label{thm:holomorphic_inverse_function_theorem}
        Let \(f: \bbC^n \to \bbC^n\) be a holomorphic function. 
        If the Jacobian determinant \(\det Df_p\) is nonzero at \(p \in \bbC^n\), then there exist open neighborhoods \(U\) of \(p\) and \(V\) of \(f(p)\) such that \(f: U \to V\) is a biholomorphism.
    \end{theorem}
    \begin{proof}
        \Yang{To be continued...}
    \end{proof}

    \begin{theorem}[Holomorphic implicit function theorem]\label{thm:holomorphic_implicit_function_theorem}
        Let \(f: \bbC^{n+m} \to \bbC^m\) be a holomorphic function. If the Jacobian determinant \(\det(\P f/\P w)\) is nonzero at \((z_0, w_0) \in \bbC^{n+m}\), then there exist open neighborhoods \(U\) of \(z_0\) and \(V\) of \(w_0\), and a unique holomorphic function \(g: U \to V\) such that for any \((z, w) \in U \times V\), \(f(z, w) = f(z_0, w_0)\) if and only if \(w = g(z)\).
        \Yang{To be continued...}
    \end{theorem}

\subsection{Cauchy Integral Formula}

    Recall the Cauchy Integral Formula in one complex variable:
    % \[ f(z) = \frac{1}{2\pi i} \int_{\partial D} \frac{f(\zeta)}{\zeta - z} d\zeta, \quad z \in D, \]
    % where \(D\) is a disk in \(\bbC\) and \(f\) is holomorphic on a neighborhood of the closure of \(D\).
    % \Yang{Need to check}

    \begin{theorem}[Cauchy Integral Formula in one complex variable]\label{thm:Cauchy_Integral_Formula_in_one_complex_variable}
        Let \(K \subset \bbC\) be a compact set with piecewise differentiable boundary \(\partial K\), and let \(f\) be differentiable on a neighborhood of \(K\). 
        Then for any \(z\) in the interior of \(K\), we have 
        \[ f(z) = \frac{1}{2\pi \im } \int_{\partial K} \frac{f(\zeta)}{\zeta - z} d\zeta + \frac{1}{2\pi \im} \int_K \frac{\P f}{\P \bar{\zeta}}(\zeta) \frac{d\bar{\zeta} \wedge d\zeta}{\zeta - z}. \]
        % \Yang{To be continued...}
    \end{theorem}
    \begin{proof}
        \Yang{By Stokes' theorem. To be continued...}
    \end{proof}

    \begin{theorem}[Cauchy Integral Formula in several complex variables]\label{thm:Cauchy_Integral_Formula_in_several_complex_variables}
        Let \(D \subset \bbC^n\) be a polydisk and \(f\) be holomorphic on a neighborhood of the closure of \(D\). 
        Then for any \(z \in D\),
        \[ f(z) = \frac{1}{(2\pi \im)^n} \int_{\partial D_1 \times \cdots \times \partial D_n} \frac{f(\zeta_1, \ldots, \zeta_n)}{(\zeta_1 - z_1) \cdots (\zeta_n - z_n)} d\zeta_1 \cdots d\zeta_n. \]
        % \Yang{To be continued...}
    \end{theorem}
    \begin{proof}
        \Yang{To be continued...}
    \end{proof}

    \begin{corollary}\label{cor:holomorphic_functions_are_analytic}
        Holomorphic functions are analytic.
        \Yang{To be continued...}
    \end{corollary}

    \begin{proposition}\label{prop:open_mapping_theorem}
        Holomorphic functions are open mappings.
        \Yang{To be continued...}
    \end{proposition}

    \begin{proposition}\label{prop:rigidity_of_holomorphic_functions}
        If a holomorphic function \(f: \Omega \to \bbC\) on a connected open set \(\Omega \subset \bbC^n\) attains its maximum at some point in \(\Omega\), then \(f\) is constant.
        \Yang{To be continued...}
    \end{proposition}

    \begin{proposition}\label{prop:Cauchy_estimates}
        Let \(D \subset \bbC^n\) be a polydisk and \(f\) be holomorphic on a neighborhood of the closure of \(D\). Then for any multi-index \(\alpha = (\alpha_1, \ldots, \alpha_n)\),
        \[ \max_{z \in D} \left| \frac{\P^{|\alpha|} f}{\P z_1^{\alpha_1} \cdots \P z_n^{\alpha_n}}(z) \right| \leq \frac{\alpha!}{r^\alpha} \max_{z \in D} |f(z)|, \]
        where \(r = (r_1, \ldots, r_n)\) is the radius of the polydisk \(D\).
        \Yang{To be continued...}
    \end{proposition}

    \begin{theorem}[Generalized Liouville Theorem]\label{thm:generalized_Liouville_Theorem}
        A holomorphic function \(f: \bbC^n \to \bbC\) on the whole space \(\bbC^n\) that satisfies a polynomial growth condition, i.e., there exist constants \(C > 0\) and \(k \geq 0\) such that 
        \[ |f(z)| \leq C(1 + |z|^k), \quad \forall z \in \bbC^n, \]
        must be a polynomial of degree at most \(k\).
        \Yang{To be continued...}
    \end{theorem}

    \begin{theorem}[Montel's Theorem]\label{thm:Montel's_Theorem}
        A family of holomorphic functions on a domain \(\Omega \subset \bbC^n\) that is uniformly bounded on compact subsets of \(\Omega\) is a normal family, i.e., every sequence in the family has a subsequence that converges uniformly on compact subsets of \(\Omega\) to a holomorphic function or to infinity.
        \Yang{To be continued...}
    \end{theorem}

\subsection{Hartogs' phenomenon}

    \begin{theorem}[Hartogs' Extension Theorem]\label{thm:Hartogs'_Extension_Theorem}
        Let \(D \subset \bbC^n\) be a domain with \(n \geq 2\), and let \(K \subset D\) be a compact subset such that \(D \setminus K\) is connected. If \(f: D \setminus K \to \bbC\) is a holomorphic function, then there exists a unique holomorphic function \(F: D \to \bbC\) such that \(F|_{D \setminus K} = f\).
        \Yang{To be continued...}
    \end{theorem}

    \begin{theorem}[Hartogs' Separate Analyticity Theorem]\label{thm:Hartogs'_Separate_Analyticity_Theorem}
        Let \(D \subset \bbC^n\) be a domain with \(n \geq 2\), and let \(f: D \to \bbC\) be a function such that for each fixed \(z' = (z_1, \ldots, z_{j-1}, z_{j+1}, \ldots, z_n)\), the function \(f(z', z_j)\) is holomorphic in \(z_j\) for all \(j = 1, \ldots, n\). Then \(f\) is holomorphic on \(D\).
        \Yang{To be continued...}
    \end{theorem}