\section{Sheaves and Bundles on Complex Manifolds}

\subsection{Fiber bundles}

    \begin{definition}\label{def:fiber_bundle_over_manifolds}
        Let \(M,F\) be manifolds.
        A \emph{fiber bundle} with fiber \(F\) over \(M\) is a surjective map \(\pi: E \to M\) of manifolds
        such that for each \(x \in M\), there exists an open neighborhood \(U\) of \(x\) and a homeomorphism
        \(\varphi: \pi^{-1}(U) \to U \times F\) such that the following diagram commutes:
        \[
            \begin{tikzcd}
                \pi^{-1}(U) \arrow{r}{\varphi} \arrow{d}{\pi} & U \times F \arrow{dl}{p_1}  \\
                U &
            \end{tikzcd}
        \]
        where \(p_1\) is the projection onto the first factor.
    \end{definition}

    Given a fiber bundle \(E\) over \(M\) with fiber \(F\) and a covering \(\{U_i\}\) of \(M\), for each \(U_i,U_j\) and \(x \in U_i \cap U_j\), we have two local trivializations
    \[
        \varphi_i|_{E_x}, \varphi_j|_{E_x}: E_x \to \{x\} \times F.
    \]
    They are differed by an automorphism \(g_{ij}(x) = \varphi_i|_{E_x} \circ (\varphi_j|_{E_x})^{-1}\) of \(\{x\} \times F\) as the following diagram
    \[ 
        \begin{tikzcd}
            \{x\} \times F \arrow[rr,"g_{ij}(x)"] & & \{x\} \times F. \\
            & E_x \arrow[lu,"\varphi_j|_{E_x}"] \arrow[ru,swap,"\varphi_i|_{E_x}"] &
        \end{tikzcd}
    \]
    The map \(g_{ij}(x)\) can be identified as an element of \(\Aut(F)\).
    Varying \(x\) in \(U_i \cap U_j\), we obtain the \emph{transition function}
    \[
        g_{ij}: U_i \cap U_j \to \Aut(F)
    \]
    which satisfies the \emph{cocycle condition}
    \[
        g_{ij}(x) \cdot g_{jk}(x) = g_{ik}(x), \quad \forall x \in U_i \cap U_j \cap U_k,
    \]
    where the multiplication \(\cdot\) is given by composition of automorphisms.

    There is a natural way to impose smooth (holomorphic) structure on \(\Aut(F)\),
    hence we can talk about smoothness or holomorphicity of transition functions.
    Set \(\Phi_{ij} = \varphi_i \circ \varphi_j^{-1}: (U_i \cap U_j) \times F \to (U_i \cap U_j) \times F\).
    Then we have \(\Phi_{ij}(x,v) = (x, g_{ij}(x)(v))\) for all \((x,v) \in (U_i \cap U_j) \times F\).
    Then \(\Phi_{ij}\) is smooth (holomorphic) if and only if \(g_{ij}\) is smooth (holomorphic).
    \Yang{To add ref.}

    Conversely, given a covering \(\{U_i\}\) of \(M\) and transition functions \(g_{ij}: U_i \cap U_j \to \Aut(F)\) satisfying the cocycle condition,
    one can glue the local trivializations \(U_i \times F\) via the maps \(\Phi_{ij}\) to obtain a fiber bundle \(E\) over \(M\) with fiber \(F\).
    Therefore, to given a fiber bundle with smooth (holomorphic) structure, it suffices to give a covering \(\{U_i\}\) of \(M\) and transition functions \(g_{ij}: U_i \cap U_j \to \Aut(F)\) which are smooth (holomorphic) and satisfy the cocycle condition.
    In general, \(\Aut(F)\) might be too large to handle.
    We can restrict the image of transition functions to a smaller subgroup \(G \subset \Aut(F)\).
    This leads to the notion of structure group.

    \begin{definition}\label{def:structure_group_of_vector_bundles}
        Let \(M,F\) be manifolds, and \(G \subset \Aut(F)\) be a Lie subgroup.
        A \emph{fiber bundle with structure group \(G\)} is a fiber bundle \(\pi: E \to M\) given by transition functions \(g_{ij}: U_i \cap U_j \to G\).
    \end{definition}

    \begin{example}\label{eg:vector_bundle_as_fiber_bundle}
        A \emph{(real) vector bundle} of rank \(r\) over a manifold \(M\) is a fiber bundle with fiber \(\bbR^r\) and structure group \(\GL_r(\bbR)\).
        Similarly, a \emph{complex vector bundle} of rank \(r\) over a manifold \(M\) is a fiber bundle with fiber \(\bbC^r\) and structure group \(\GL_r(\bbC)\).

        On a real manifold \(M\) of dimension \(2n\), an almost complex structure is equivalent to a reduction of the structure group of the tangent bundle \(TM\) from \(\GL_{2n}(\bbR)\) to \(\GL_n(\bbC)\).
    \end{example}

    By the transition functions construction, we can see that 
    \begin{theorem}\label{thm:classification_fiber_bundle_by_sheaves_cohomology}
        Let \(M,F\) be locally ringed spaces and \(G \subset \Aut(F)\) a subgroup.
        Set \(\calG\) be the sheaf of ``admissible'' functions from open subsets of \(M\) to \(G\).
        Then the set of isomorphism classes of fiber bundles over \(M\) with fiber \(F\) and structure group \(G\) is in one-to-one correspondence with the \v{C}ech cohomology set \(\check{H}^1(M,\calG)\).
    \end{theorem}

    For example, if \(F = \bbC\) and \(G = \GL_1(\bbC) = \bbC^*\), consider the holomorphic line bundles over a complex manifold \(M\).
    The sheaf \(\calG\) is equal to \(\calO_M^*\), the sheaf of nowhere vanishing holomorphic functions on \(M\).
    Therefore, by \cref{thm:classification_fiber_bundle_by_sheaves_cohomology}, we get the classic result \(\Pic(M) \cong \check{H}^1(M, \calO_M^*)\).

    \begin{slogan}\label{slogan:things_of_fiber_bundles_we_care_about}
        For a fiber bundle \(E\) over \(M\), we care about
        \begin{itemize}
            \item fiber \(F\),
            \item structure group \(G \subset \Aut(F)\),
            \item ``admissible'' functions class of transition functions \(g_{ij}: U_i \cap U_j \to G\) (e.g. continuous, smooth, holomorphic).
        \end{itemize}
    \end{slogan}
    

\subsection{Sheaves}


    \begin{construction}\label{cons:sheaf_of_sections_of_bundle}
        Let \(M\) be a manifold and \(\pi: E \to M\) be a fiber bundle with fiber \(F\).
        For each open subset \(U \subset M\), we can consider the set of "admissible" sections of \(E\) over \(U\):
        \[ 
            \Gamma(U,E) = \{ s: U \to E \mid \pi \circ s = \id_U, s \text{ is admissible} \}.
        \]
        Here ``admissible'' means continuous, smooth, holomorphic, etc., depending on the context.
        The assignment \(U \mapsto \Gamma(U,E)\) defines a sheaf of sets (or groups, modules, etc. if \(F\) has additional structure) on \(M\), called the \emph{sheaf of sections} of the bundle \(E\).
    \end{construction}

    \begin{example}\label{eg:tangent_and_cotangent_bundle_as_sheaves}
        Let \(M\) be a complex manifold.
        We explain how to view the tangent bundle \(TM\) and the cotangent bundle \(T^*M\) as sheaves.
        There are two important classes of admissible sections of these bundles, namely holomorphic and smooth sections.
        We denote the sheaf of holomorphic (respectively smooth) sections of \(TM\) by \(\calT_{M,\hol}\) (respectively \(\calT_{M,\sm}\)).

        Correspondingly, we denote the sheaf of holomorphic (respectively smooth) sections of \(T^*M\) by \(\Omega_{M,\hol}^1\) (respectively \(\Omega_{M,\sm}^1\)).
        Sometime we omit the subscript \(M\) if there is no confusion.

        
        The elements in \(\calT_{M,\hol}(U)\) (respectively \(\calT_{M,\sm}(U)\)) are holomorphic (respectively smooth) vector fields on \(U\),
        and holomorphic (respectively smooth) \(1\)-forms on \(U\) for \(\Omega_{M,\hol}^1(U)\) (respectively \(\Omega_{M,\sm}^1(U)\)).
        As sheaves, we have 
        \[ \Omega_{M,\hol}^1 \cong \calHom_{\calO_M}(\calT_{M,\hol}, \calO_M) \quad \text{and} \quad \Omega_{M,\sm}^1 \cong \calHom_{\calC_M^\infty}(\calT_{M,\sm}, \calC_M^\infty), \]
        where \(\calC_M^\infty\) is the sheaf of smooth complex-valued functions on \(M\).
    \end{example}


    % \Yang{To be checked.}
    For \(T^*M^\bbC := T^*M \otimes_\bbR \bbC\), this is also a holomorphic complex vector bundle over \(M\).
    Hence we have the sheaf of holomorphic sections \(\Omega_{M,\hol}^\bbC\).
    However, in general \(T^*M^{0,1}\) is not a holomorphic bundle since its transition functions are given by the complex conjugates of those of \(T^*M^{1,0} \cong T^*M\).
    Hence for the decomposition \(T^*M^\bbC = T^*M^{1,0} \oplus T^*M^{0,1}\), we only have the decomposition of sheaves of smooth sections
    \[ \Omega_{\sm}^{1,\bbC} \cong \Omega_{\sm}^{1,0} \oplus \Omega_{\sm}^{0,1} \cong \Omega_{\sm}^1 \oplus \overline{\Omega_{\sm}^1}. \]

    
