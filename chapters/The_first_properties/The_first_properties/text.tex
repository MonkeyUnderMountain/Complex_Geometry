\section{The first properties}


\subsection{Holomorphic functions}

    We identify \(\bbC^n \cong \bbR^{2n}\).

    \begin{definition}\label{def:differentiable_and_holomorphic_function}
        A function \(f: \bbR^{2n} \to \bbR^{2m}\) is \emph{differentiable} at \(p \in \bbR^{2n}\) if there exists a linear map \(L: \bbR^{2n} \to \bbR^{2m}\) such that 
        \[ f(z) = f(p) + L(z - p) + o(|z - p|). \]
        A function \(f: \bbC^n \to \bbC^m\) is \emph{holomorphic} at \(p \in \bbC^n\) if it is differentiable at \(p\) and \Yang{To be continued...}
    \end{definition}

    \begin{definition}\label{def:Wirtinger_operators}
        The \emph{Wirtinger operators} are defined as 
        \[ \P{z_j} = \frac{1}{2}\left( \P{}{x_j} - i\P{}{y_j} \right), \quad \P{\bar{z}_j} = \frac{1}{2}\left( \P{}{x_j} + i\P{}{y_j} \right). \]
    \end{definition}

    Then we can rewrite the Cauchy-Riemann equations as
    \[ \P{\bar{z}_j} f = 0, \quad j = 1, \ldots, n. \]

    We summarize some properties of Wirtinger operators in the following proposition.
    \begin{proposition}\label{prop:properties_of_Wirtinger_operators}
        The Wirtinger operators satisfy the following properties:
        \begin{enumerate}
            \item \(\P{z_j}\) and \(\P{\bar{z}_j}\) are linear operators.
            \item \(\P{z_j} z_k = \delta_{jk}\), \(\P{z_j} \bar{z}_k = 0\), \(\P{\bar{z}_j} z_k = 0\), and \(\P{\bar{z}_j} \bar{z}_k = \delta_{jk}\).
            \item The Leibniz rule holds: for any two functions \(f, g: \bbC^n \to \bbC\),
            \[ \P{z_j}(fg) = (\P{z_j} f)g + f(\P{z_j} g), \quad \P{\bar{z}_j}(fg) = (\P{\bar{z}_j} f)g + f(\P{\bar{z}_j} g). \]
            \item The operators commute: for any \(j, k\),
            \[ \P{z_j}\P{\bar{z}_k} = \P{\bar{z}_k}\P{z_j}, \quad \P{z_j}\P{z_k} = \P{z_k}\P{z_j}, \quad \P{\bar{z}_j}\P{\bar{z}_k} = \P{\bar{z}_k}\P{\bar{z}_j}. \]
        \end{enumerate}
        \Yang{To be continued...}
    \end{proposition}

    Consider the complexified differential \(T\bbR^{2n} \otimes_\bbR \bbC\). 
    We can extend the Wirtinger operators to complexified tangent vectors by linearity. 
    Then \(\P_{z_j}\) and \(\P{\bar{z}_j}\) form a basis of \(T^*\bbR^{2n} \otimes_\bbR \bbC\).
    If \(f\) is holomorphic, under this basis, its differential \(df\) can be written as
    \[ df = \begin{pmatrix}
        \frac{\P f}{\P z} & \\
        & \overline{\frac{\P f}{\P z}}
    \end{pmatrix} \].

    \begin{theorem}[Holomorphic inverse function theorem]\label{thm:holomorphic_inverse_function_theorem}
        Let \(f: \bbC^n \to \bbC^n\) be a holomorphic function. If the Jacobian determinant \(\det(\P f/\P z)\) is nonzero at \(p \in \bbC^n\), then there exist open neighborhoods \(U\) of \(p\) and \(V\) of \(f(p)\) such that \(f: U \to V\) is a biholomorphism.
        \Yang{To be continued...}
    \end{theorem}

    \begin{theorem}[Holomorphic implicit function theorem]\label{thm:holomorphic_implicit_function_theorem}
        Let \(f: \bbC^{n+m} \to \bbC^m\) be a holomorphic function. If the Jacobian determinant \(\det(\P f/\P w)\) is nonzero at \((z_0, w_0) \in \bbC^{n+m}\), then there exist open neighborhoods \(U\) of \(z_0\) and \(V\) of \(w_0\), and a unique holomorphic function \(g: U \to V\) such that for any \((z, w) \in U \times V\), \(f(z, w) = f(z_0, w_0)\) if and only if \(w = g(z)\).
        \Yang{To be continued...}
    \end{theorem}

\subsection{Cauchy Integral Formula}

    Recall the Cauchy Integral Formula in one complex variable:
    \[ f(z) = \frac{1}{2\pi i} \int_{\partial D} \frac{f(\zeta)}{\zeta - z} d\zeta, \quad z \in D, \]
    where \(D\) is a disk in \(\bbC\) and \(f\) is holomorphic on a neighborhood of the closure of \(D\).
    \Yang{Need to check}

    \begin{theorem}[Cauchy Integral Formula in one complex variable]\label{thm:Cauchy_Integral_Formula_in_one_complex_variable}
        Let \(D \subset \bbC\) be a disk and \(f\) be holomorphic on a neighborhood of the closure of \(D\). Then for any \(z \in D\),
        \[ f(z) = \frac{1}{2\pi i} \int_{\partial D} \frac{f(\zeta)}{\zeta - z} d\zeta. \]
        \Yang{To be continued...}
    \end{theorem}

    \begin{theorem}[Cauchy Integral Formula in several complex variables]\label{thm:Cauchy_Integral_Formula_in_several_complex_variables}
        Let \(D \subset \bbC^n\) be a polydisk and \(f\) be holomorphic on a neighborhood of the closure of \(D\). Then for any \(z \in D\),
        \[ f(z) = \frac{1}{(2\pi i)^n} \int_{\partial D_1 \times \cdots \times \partial D_n} \frac{f(\zeta_1, \ldots, \zeta_n)}{(\zeta_1 - z_1) \cdots (\zeta_n - z_n)} d\zeta_1 \cdots d\zeta_n. \]
        \Yang{To be continued...}
    \end{theorem}

    \begin{corollary}\label{cor:holomorphic_functions_are_analytic}
        Holomorphic functions are analytic.
        \Yang{To be continued...}
    \end{corollary}

    \begin{proposition}\label{prop:open_mapping_theorem}
        Holomorphic functions are open mappings.
        \Yang{To be continued...}
    \end{proposition}

    \begin{proposition}\label{prop:rigidity_of_holomorphic_functions}
        If a holomorphic function \(f: \Omega \to \bbC\) on a connected open set \(\Omega \subset \bbC^n\) attains its maximum at some point in \(\Omega\), then \(f\) is constant.
        \Yang{To be continued...}
    \end{proposition}

    \begin{proposition}\label{prop:Cauchy_estimates}
        Let \(D \subset \bbC^n\) be a polydisk and \(f\) be holomorphic on a neighborhood of the closure of \(D\). Then for any multi-index \(\alpha = (\alpha_1, \ldots, \alpha_n)\),
        \[ \max_{z \in D} \left| \frac{\P^{|\alpha|} f}{\P z_1^{\alpha_1} \cdots \P z_n^{\alpha_n}}(z) \right| \leq \frac{\alpha!}{r^\alpha} \max_{z \in D} |f(z)|, \]
        where \(r = (r_1, \ldots, r_n)\) is the radius of the polydisk \(D\).
        \Yang{To be continued...}
    \end{proposition}

    \begin{theorem}[Generalized Liouville Theorem]\label{thm:generalized_Liouville_Theorem}
        A holomorphic function \(f: \bbC^n \to \bbC\) on the whole space \(\bbC^n\) that satisfies a polynomial growth condition, i.e., there exist constants \(C > 0\) and \(k \geq 0\) such that 
        \[ |f(z)| \leq C(1 + |z|^k), \quad \forall z \in \bbC^n, \]
        must be a polynomial of degree at most \(k\).
        \Yang{To be continued...}
    \end{theorem}

    \begin{theorem}[Montel's Theorem]\label{thm:Montel's_Theorem}
        A family of holomorphic functions on a domain \(\Omega \subset \bbC^n\) that is uniformly bounded on compact subsets of \(\Omega\) is a normal family, i.e., every sequence in the family has a subsequence that converges uniformly on compact subsets of \(\Omega\) to a holomorphic function or to infinity.
        \Yang{To be continued...}
    \end{theorem}

\subsection{Hartogs' phenomenon}

    \begin{theorem}[Hartogs' Extension Theorem]\label{thm:Hartogs'_Extension_Theorem}
        Let \(D \subset \bbC^n\) be a domain with \(n \geq 2\), and let \(K \subset D\) be a compact subset such that \(D \setminus K\) is connected. If \(f: D \setminus K \to \bbC\) is a holomorphic function, then there exists a unique holomorphic function \(F: D \to \bbC\) such that \(F|_{D \setminus K} = f\).
        \Yang{To be continued...}
    \end{theorem}

    \begin{theorem}[Hartogs' Separate Analyticity Theorem]\label{thm:Hartogs'_Separate_Analyticity_Theorem}
        Let \(D \subset \bbC^n\) be a domain with \(n \geq 2\), and let \(f: D \to \bbC\) be a function such that for each fixed \(z' = (z_1, \ldots, z_{j-1}, z_{j+1}, \ldots, z_n)\), the function \(f(z', z_j)\) is holomorphic in \(z_j\) for all \(j = 1, \ldots, n\). Then \(f\) is holomorphic on \(D\).
        \Yang{To be continued...}
    \end{theorem}
